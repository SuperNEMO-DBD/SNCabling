\section{Addressing objects}

\subsection{Format of a CaloSignal label}

Each label used for calorimeter signal cabling uses  the following
format:

\begin{center}
  \fbox{\texttt{X:}$id_1$.$id_2$\dots$id_n$}
\end{center}
\noindent  where \texttt{X}  is a  single letter  which describes  the
category  of the  labelled  object,  and the  $id_1$.$id_2$\dots$id_n$
sequence is the unique address of the object within its category.  The
$id_x$ tokens are positive integers (possibly zero).  The \emph{colon}
character is  used to separate  the category letter from  the address.
The  sequence  of  identifiers  in  the  address  use  the  \emph{dot}
character as a separator.


\subsection{Front-end crates, boards and channels}

Front-end crates belonging  to the CaloSignal system  are installed in
racks  number  2 and  3  on  the  electronics  platform.  A  crate  is
identified with an unique ID,  namely a number ranging from \texttt{0}
to \texttt{2}.   A given  crate is  identified by  its label  with the
following scheme:
\begin{center}
   \fbox{\texttt{C:$crate$}}
 \end{center}
where \texttt{$crate$} is the number of the crate (positive integer).
\vskip     10pt    \par\noindent     Examples:    \fbox{\texttt{C:0}},
\fbox{\texttt{C:1}},
\fbox{\texttt{C:2}}.   \par Conventionally,  crate \texttt{0}  manages
PMTs from  the main  wall on the  \emph{Italy} side,  crate \texttt{1}
manages PMTs  from the main wall  on the \emph{France} side  and crate
\texttt{2} manages PMTs from X-walls and gamma veto.

 \vskip 10pt A crate contains up to twenty 16-channel boards.  A board
 inherits the  number of the crate  it is plugged into  and is located
 through its  slot number. A  given board  is identified by  its label
 with the following scheme:
\begin{center}
   \fbox{\texttt{B:$crate$.$board$}}
 \end{center}
where \texttt{$crate$} is the number of the crate and \texttt{$board$}
is  the  number  of  the  board  (slot)  ranging  from  \texttt{0}  to
\texttt{20}  except  number  \texttt{10}  which is  reserved  for  the
control board slot number.
\vskip 10pt
\par\noindent Examples: \fbox{\texttt{B:0.0}},  \dots
\fbox{\texttt{B:1.11}}.

\vskip 10pt  \par\noindent Up  to 16 channels  are addressed  within a
board.  A given channel is identified  by its label with the following
scheme:
\begin{center}
   \fbox{\texttt{H:$crate$.$board$.$channel$}}
 \end{center}
where \texttt{$crate$} is the number of the crate, \texttt{$board$} is
the number  of the board  in the  crate and \texttt{$channel$}  is the
number  of  the  channel  in  the board  ranging  from  \texttt{0}  to
\texttt{15}.   \par\noindent Examples:  \fbox{\texttt{H:0.0.0}}, \dots
\fbox{\texttt{H:1.19.12}}

\vskip 10pt A channel is  automatically associated to a specific cable
number  in  a given  harness.  The  association between  each  readout
channel and the cable it is connected to within a harness is described
by a dedicated cabling table.

\subsection{Harnesses and cables}

One uses 2$\times$22  harnesses to group individual  channels from the
front-end boards to the PMTs.

\vskip 10pt  An external signal  harness contains a set  of individual
external  signal  cables  that  connect  to the  patch  panel  and  to
individual  channels  divided up  into  several  frontend boards.   An
external signal  harness has an  unique ID ranging from  \texttt{0} to
\texttt{21}.  An external signal harness is identified by a label with
the following scheme:
\begin{center}
   \fbox{\texttt{E:$harness$}}
 \end{center}
where \texttt{$harness$} is the number of the external harness.
\par\noindent Examples: \fbox{\texttt{E:0}}, \dots
\fbox{\texttt{E:21}}

\vskip 10pt  An internal signal  harness contains a set  of individual
internal signal cables that connect to the patch panel and a given set
of PMTs.  An internal  signal harness  has an  unique ID  ranging from
\texttt{0} to  \texttt{21}.  An internal signal  harness is identified
by a label with the following scheme:
\begin{center}
   \fbox{\texttt{I:$harness$}}
 \end{center}
where \texttt{$harness$} is the number of the internal harness.
\par\noindent Examples: \fbox{\texttt{I:0}}, \dots
\fbox{\texttt{I:21}}
\vskip 10pt An internal harness with a given number is associated to a
external harness  with the  same number.


\vskip 10pt Each  external or internal harness groups up  to 40 signal
cables identified through  a cable ID from  \texttt{0} to \texttt{39}.
An external signal  cable is identified by a label  with the following
scheme:
\begin{center}
   \fbox{\texttt{L:$harness$.$cable$}}
 \end{center}
where \texttt{$harness$}  is the  number of  the external  harness and
\texttt{$cable$} is  the number of the  external cable.  \par\noindent
Examples: \fbox{\texttt{L:0.0}}, \dots \fbox{\texttt{L:21.15}}.

\vskip 10pt Internal signal cables use a similar scheme:
\begin{center}
   \fbox{\texttt{A:$harness$.$cable$}}
\end{center}
\par\noindent Examples: \fbox{\texttt{A:0.0}}, \dots
\fbox{\texttt{A:21.15}}.

Internal and external signal cables  connected through the patch panel
(MCX male/female connectors) share the same harness and cable numbers.
Thus, the  internal cable  \fbox{\texttt{A:7.12}} is connected  to the
external cable \fbox{\texttt{L:7.12}}:
\begin{center}
\fbox{\texttt{L:7.12} \texttt{->} \texttt{A:7.12}}.
\end{center}


\subsection{Optical modules}


The  identification scheme  of the  optical  modules and their PMTs
is  based on  the
addressing scheme defined in the geometry model and implemented in the
simulation     and     data    analysis     software\footnote{Falaise:
  \url{https://gitub.com/SuperNEMO-DBD/Falaise}}.     There   are    4
categories  of  optical  modules  and  thus  of  scintillator  blocks,
depending on their location in the experimental setup:

\begin{itemize}
  
\item Main wall block (Falaise: geometry category \texttt{"calorimeter\_block"}
  and  type  \texttt{1302}):

  \par  OMs are  addressed  through  their
  \emph{side}   number from 0 (Italy)     to  1 (France),
  \emph{column} number from 0 (Edelweiss) to 19 (Tunnel) and
  \emph{row}    number from 0 (bottom)    to 12 (top).

  \par We  propose to label such a block with the following scheme:
  \begin{center}
    \fbox{\texttt{M:$side$.$column$.$row$}}
  \end{center}
  \vskip 10pt
  \par\noindent Examples: \fbox{\texttt{M:0.0.0}}, 
  \fbox{\texttt{M:0.19.12}}, \fbox{\texttt{M:1.0.0}}, \fbox{\texttt{M:1.19.12}}.
  
\item X-wall block (Falaise: geometry category \texttt{"xcalo\_block"} and type
  \texttt{1232}):
  
  \par OMs  are addressed  through their
  \emph{side}   number from 0 (Italy)     to  1 (France),
  \emph{wall}   number from 0 (Edelweiss) to  1 (tunnel),
  \emph{column} number from 0 (source)    to  1 (calorimeter) and
  \emph{row}    number from 0 (bottom)    to 15 (top).
  
  \par  We propose  to
  label such a block with the following scheme:
  \begin{center}
    \fbox{\texttt{X:$side$.$wall$.$column$.$row$}}
  \end{center}
  \vskip 10pt
  \par\noindent Examples: \fbox{\texttt{X:0.1.1.15}}, \fbox{\texttt{X:1.0.0.8}}
  
\item Gamma veto block  (Falaise: geometry category \texttt{"gveto\_block"} and
  type   \texttt{1252}):
  
  \par   OMs  are   addressed  through   their
  \emph{side}   number from 0 (Italy)     to  1 (France),
  \emph{wall}   number from 0 (bottom)    to  1 (top)  and
  \emph{column} number from 0 (Edelweiss) to 15 (tunnel).
  
  \par We propose to label  such a block
  with the following scheme:
  \begin{center}
    \fbox{\texttt{G:$side$.$wall$.$column$}}
  \end{center}
  \vskip 10pt
  
  \par\noindent Examples: \fbox{\texttt{G:0.1.0}}, \fbox{\texttt{G:1.0.8}}
  
\item  Block for  reference  optical module:
  
  \par  OMs are  addressed  through their \emph{ref}  number.
  
  \par We propose to  label such a  block with the following scheme:
  \begin{center}
    \fbox{\texttt{R:$ref$}}
  \end{center}

\end{itemize}


\subsection{Summary}

\begin{center}
\begin{tabular}{|c|c|c|}
  \hline
  Category & Symbol & Depth \\
  \hline
  \hline
  \textbf{C}rate & \texttt{C} & 1 \\
  \hline
  \textbf{B}oard & \texttt{B} & 2 \\
  \hline
  C\textbf{h}annel & \texttt{H} & 3 \\
  \hline
  \textbf{E}xternal harness & \texttt{E} & 1 \\
  \hline
  External cab\textbf{l}e & \texttt{L} & 2 \\
  \hline
  \textbf{I}nternal harness & \texttt{I} & 1 \\
  \hline
  Internal c\textbf{A}ble & \texttt{A} & 2 \\
  \hline
  \textbf{M}ain wall PMT & \texttt{M} & 3 \\
  \hline
  \textbf{X}-wall PMT & \texttt{X} & 4 \\
  \hline
  \textbf{G}amma veto PMT & \texttt{G} & 3 \\
  \hline
\end{tabular}
\end{center}

%% end
