\documentclass[12pt,a4paper]{article}

\usepackage{a4wide}
\usepackage{verbatim}
\usepackage[T1]{fontenc}
\usepackage{ucs}
\usepackage[utf8x]{inputenc}
\usepackage[dvips]{color}
\usepackage{graphicx}
\usepackage{epic}
\usepackage{eepic}
\usepackage{eepicemu}
\usepackage{array}
\usepackage{moreverb}
\usepackage{fancyvrb}
\usepackage{url}
\usepackage{eurosym}
\usepackage{amsmath}
\usepackage{amssymb}
\usepackage{multicol}

\newcommand{\imgpath}{./images}
\newcommand{\samplespath}{./samples}
\newcommand{\pdfteximgpath}{./pdftex}
\newcommand{\pdftextimgpath}{./pdftex_t}

\title{SuperNEMO Demonstrator\\
  Calorimeter Signal Readout System (CaloSignal)\\
  Cabling scheme and cable labels\\
  version 0.1}
\author{M.Bongrand, Y.Lemi\`ere, F.Mauger}
\date{November 3rd, 2018}

%%%%%%%%%%%%%%%%
\begin{document}

\maketitle

\begin{abstract}
  \noindent This document presents the cable labelling convention used
  for  the SuperNEMO  Demonstrator's Calorimeter Signal Readout  System
  (CaloSignal).  We reuse  here some informations available  from a couple
  of documents prepared  by Cedric and Mathieu, SNDER,  with some adaptations,
  fixes and addons.
  
  \vskip 10pt
  \noindent This document and all associated tools
  are hosted at:
  \vskip 5pt
  \url{https://gitlab.in2p3.fr/SuperNEMO-DBD/SNCabling}.

\end{abstract}

\tableofcontents
\vfill

\clearpage
\section{Principle}

The  SuperNEMO   Demonstrator's  Calorimeter  Signal   Readout  System
(CaloSignal)  uses three  crates which  host a  total of  52 front-end
boards to collect signals from 712 PMTs.

Each front end board  manages up to 16 channels and  is connected to a
specific set  of PMTs through a  pair of harnesses.  A  first harness,
called \emph{external harness},  groups a set of  signal cables linked
to the Wavecatcher front-end board  through MCX male connectors and to
the  patch panel  (external side)  also through  MCX male  connectors.
From  the internal side  of patch panel, a
new  harness,  namely  the \emph{internal  harness},  routes  internal
signal cables to  the PMTs.  Both ends of  the \emph{internal harness}
are designed in  such  a  way internal  signal  cables  can be  connected
individually from  the patch  panel to their  associated PMTs.  On the
patch panel  side, an  internal signal  cable is  equipped with  a MCX
female  connector.  On  the  PMT  side, an  internal  signal cable  is
equipped with  a Souriau pin.  A  given harness handles a  set of PMTs
that are geographically close to each  other, in order to optimize the
length of the cables.

In  the present  scheme, it  has been  decided to  identify individual
signal cables within a harness  using a unique cable identifier.  This
identifier is  propagated through the  patch panel  to the end  of the
cable  linked to  the PMT.   This scheme  implies to  use a  dedicated
cabling table.

\noindent\par Figure \ref{fig:calosignal:principle:1}  shows the basic
diagram of the CaloSignal system. A dedicated labelling system is used
to ease the cabling operations (see next sections).

\begin{figure}[h!]
  \begin{center}
    \scalebox{0.75}{\input{\pdftextimgpath/fig-calosignal-1.pdftex_t}}
  \end{center}
  \caption{Principle of the  signal distribution from the  PMTs to the
    Wavecatcher CaloFEB boards and labelling of cables and harnesses.}
  \label{fig:calosignal:principle:1}
\end{figure}

\clearpage
\section{Addressing objects}

\subsection{Format of a CaloSignal label}

Each label used for calorimeter signal cabling uses  the following
format:

\begin{center}
  \fbox{\texttt{X:}$id_1$.$id_2$\dots$id_n$}
\end{center}
\noindent  where \texttt{X}  is a  single letter  which describes  the
category  of the  labelled  object,  and the  $id_1$.$id_2$\dots$id_n$
sequence is the unique address of the object within its category.  The
$id_x$ tokens are positive integers (possibly zero).  The \emph{colon}
character is  used to separate  the category letter from  the address.
The  sequence  of  identifiers  in  the  address  use  the  \emph{dot}
character as a separator.


\subsection{Front-end crates, boards and channels}

Front-end crates belonging  to the CaloSignal system  are installed in
racks  number  2 and  3  on  the  electronics  platform.  A  crate  is
identified with an unique ID,  namely a number ranging from \texttt{0}
to \texttt{2}.   A given  crate is  identified by  its label  with the
following scheme:
\begin{center}
   \fbox{\texttt{C:$crate$}}
 \end{center}
where \texttt{$crate$} is the number of the crate (positive integer).
\vskip     10pt    \par\noindent     Examples:    \fbox{\texttt{C:0}},
\fbox{\texttt{C:1}},
\fbox{\texttt{C:2}}.   \par Conventionally,  crate \texttt{0}  manages
PMTs from  the main  wall on the  \emph{Italy} side,  crate \texttt{1}
manages PMTs  from the main wall  on the \emph{France} side  and crate
\texttt{2} manages PMTs from X-walls and gamma veto.

 \vskip 10pt A crate contains up to twenty 16-channel boards.  A board
 inherits the  number of the crate  it is plugged into  and is located
 through its  slot number. A  given board  is identified by  its label
 with the following scheme:
\begin{center}
   \fbox{\texttt{B:$crate$.$board$}}
 \end{center}
where \texttt{$crate$} is the number of the crate and \texttt{$board$}
is  the  number  of  the  board  (slot)  ranging  from  \texttt{0}  to
\texttt{20}  except  number  \texttt{10}  which is  reserved  for  the
control board slot number.
\vskip 10pt
\par\noindent Examples: \fbox{\texttt{B:0.0}},  \dots
\fbox{\texttt{B:1.11}}.

\vskip 10pt  \par\noindent Up  to 16 channels  are addressed  within a
board.  A given channel is identified  by its label with the following
scheme:
\begin{center}
   \fbox{\texttt{H:$crate$.$board$.$channel$}}
 \end{center}
where \texttt{$crate$} is the number of the crate, \texttt{$board$} is
the number  of the board  in the  crate and \texttt{$channel$}  is the
number  of  the  channel  in  the board  ranging  from  \texttt{0}  to
\texttt{15}.   \par\noindent Examples:  \fbox{\texttt{H:0.0.0}}, \dots
\fbox{\texttt{H:1.19.12}}

\vskip 10pt A channel is  automatically associated to a specific cable
number  in  a given  harness.  The  association between  each  readout
channel and the cable it is connected to within a harness is described
by a dedicated cabling table.

\subsection{Harnesses and cables}

One uses 2$\times$22  harnesses to group individual  channels from the
front-end boards to the PMTs.

\vskip 10pt  An external signal  harness contains a set  of individual
external  signal  cables  that  connect  to the  patch  panel  and  to
individual  channels  divided up  into  several  frontend boards.   An
external signal  harness has an  unique ID ranging from  \texttt{0} to
\texttt{21}.  An external signal harness is identified by a label with
the following scheme:
\begin{center}
   \fbox{\texttt{E:$harness$}}
 \end{center}
where \texttt{$harness$} is the number of the external harness.
\par\noindent Examples: \fbox{\texttt{E:0}}, \dots
\fbox{\texttt{E:21}}

\vskip 10pt  An internal signal  harness contains a set  of individual
internal signal cables that connect to the patch panel and a given set
of PMTs.  An internal  signal harness  has an  unique ID  ranging from
\texttt{0} to  \texttt{21}.  An internal signal  harness is identified
by a label with the following scheme:
\begin{center}
   \fbox{\texttt{I:$harness$}}
 \end{center}
where \texttt{$harness$} is the number of the internal harness.
\par\noindent Examples: \fbox{\texttt{I:0}}, \dots
\fbox{\texttt{I:21}}
\vskip 10pt An internal harness with a given number is associated to a
external harness  with the  same number.


\vskip 10pt Each  external or internal harness groups up  to 40 signal
cables identified through  a cable ID from  \texttt{0} to \texttt{39}.
An external signal  cable is identified by a label  with the following
scheme:
\begin{center}
   \fbox{\texttt{L:$harness$.$cable$}}
 \end{center}
where \texttt{$harness$}  is the  number of  the external  harness and
\texttt{$cable$} is  the number of the  external cable.  \par\noindent
Examples: \fbox{\texttt{L:0.0}}, \dots \fbox{\texttt{L:21.15}}.

\vskip 10pt Internal signal cables use a similar scheme:
\begin{center}
   \fbox{\texttt{A:$harness$.$cable$}}
\end{center}
\par\noindent Examples: \fbox{\texttt{A:0.0}}, \dots
\fbox{\texttt{A:21.15}}.

Internal and external signal cables  connected through the patch panel
(MCX male/female connectors) share the same harness and cable numbers.
Thus, the  internal cable  \fbox{\texttt{A:7.12}} is connected  to the
external cable \fbox{\texttt{L:7.12}}:
\begin{center}
\fbox{\texttt{L:7.12} \texttt{->} \texttt{A:7.12}}.
\end{center}


\subsection{Optical modules}


The  identification scheme  of the  optical  modules and their PMTs
is  based on  the
addressing scheme defined in the geometry model and implemented in the
simulation     and     data    analysis     software\footnote{Falaise:
  \url{https://gitub.com/SuperNEMO-DBD/Falaise}}.     There   are    4
categories  of  optical  modules  and  thus  of  scintillator  blocks,
depending on their location in the experimental setup:

\begin{itemize}
  
\item Main wall block (Falaise: geometry category \texttt{"calorimeter\_block"}
  and  type  \texttt{1302}):

  \par  OMs are  addressed  through  their
  \emph{side}   number from 0 (Italy)     to  1 (France),
  \emph{column} number from 0 (Edelweiss) to 19 (Tunnel) and
  \emph{row}    number from 0 (bottom)    to 12 (top).

  \par We  propose to label such a block with the following scheme:
  \begin{center}
    \fbox{\texttt{M:$side$.$column$.$row$}}
  \end{center}
  \vskip 10pt
  \par\noindent Examples: \fbox{\texttt{M:0.0.0}}, 
  \fbox{\texttt{M:0.19.12}}, \fbox{\texttt{M:1.0.0}}, \fbox{\texttt{M:1.19.12}}.
  
\item X-wall block (Falaise: geometry category \texttt{"xcalo\_block"} and type
  \texttt{1232}):
  
  \par OMs  are addressed  through their
  \emph{side}   number from 0 (Italy)     to  1 (France),
  \emph{wall}   number from 0 (Edelweiss) to  1 (tunnel),
  \emph{column} number from 0 (source)    to  1 (calorimeter) and
  \emph{row}    number from 0 (bottom)    to 15 (top).
  
  \par  We propose  to
  label such a block with the following scheme:
  \begin{center}
    \fbox{\texttt{X:$side$.$wall$.$column$.$row$}}
  \end{center}
  \vskip 10pt
  \par\noindent Examples: \fbox{\texttt{X:0.1.1.15}}, \fbox{\texttt{X:1.0.0.8}}
  
\item Gamma veto block  (Falaise: geometry category \texttt{"gveto\_block"} and
  type   \texttt{1252}):
  
  \par   OMs  are   addressed  through   their
  \emph{side}   number from 0 (Italy)     to  1 (France),
  \emph{wall}   number from 0 (bottom)    to  1 (top)  and
  \emph{column} number from 0 (Edelweiss) to 15 (tunnel).
  
  \par We propose to label  such a block
  with the following scheme:
  \begin{center}
    \fbox{\texttt{G:$side$.$wall$.$column$}}
  \end{center}
  \vskip 10pt
  
  \par\noindent Examples: \fbox{\texttt{G:0.1.0}}, \fbox{\texttt{G:1.0.8}}
  
\item  Block for  reference  optical module:
  
  \par  OMs are  addressed  through their \emph{ref}  number.
  
  \par We propose to  label such a  block with the following scheme:
  \begin{center}
    \fbox{\texttt{R:$ref$}}
  \end{center}

\end{itemize}


\subsection{Summary}

\begin{center}
\begin{tabular}{|c|c|c|}
  \hline
  Category & Symbol & Depth \\
  \hline
  \hline
  \textbf{C}rate & \texttt{C} & 1 \\
  \hline
  \textbf{B}oard & \texttt{B} & 2 \\
  \hline
  C\textbf{h}annel & \texttt{H} & 3 \\
  \hline
  \textbf{E}xternal harness & \texttt{E} & 1 \\
  \hline
  External cab\textbf{l}e & \texttt{L} & 2 \\
  \hline
  \textbf{I}nternal harness & \texttt{I} & 1 \\
  \hline
  Internal c\textbf{A}ble & \texttt{A} & 2 \\
  \hline
  \textbf{M}ain wall PMT & \texttt{M} & 3 \\
  \hline
  \textbf{X}-wall PMT & \texttt{X} & 4 \\
  \hline
  \textbf{G}amma veto PMT & \texttt{G} & 3 \\
  \hline
\end{tabular}
\end{center}

%% end


\clearpage
\section{The Calorimeter crates}

Figures  \ref{fig:calosignal:crates:0},  \ref{fig:calosignal:crates:1}
and \ref{fig:calosignal:crates:2} show the association of the external
harnesses  with the  Wavecatcher front-end  board channels  within the
calorimeter front-end crates  0 (Italy), 1 (France) and  2 (X-wall and
gamma veto).

\begin{figure}[h!]
  \begin{center}
    \scalebox{0.6}{\input{\pdftextimgpath/fig-calosignal-crate-0.pdftex_t}}
  \end{center}
  \caption{Calorimeter front-end crate 0 (main wall Italy).}
  \label{fig:calosignal:crates:0}
\end{figure}

\begin{figure}[h!]
  \begin{center}
    \scalebox{0.6}{\input{\pdftextimgpath/fig-calosignal-crate-1.pdftex_t}}
  \end{center}
  \caption{Calorimeter front-end crate 1 (main wall France).}
  \label{fig:calosignal:crates:1}
 
\end{figure}

\begin{figure}[h!]
  \begin{center}
    \scalebox{0.6}{\input{\pdftextimgpath/fig-calosignal-crate-2.pdftex_t}}
  \end{center}
  \caption{Calorimeter front-end crate 2 (X-wall and gamma veto).}
  \label{fig:calosignal:crates:2}
 
\end{figure}

Figure  \ref{fig:calosignal:crates:3} shows the path of the
external signal cables and harnesses from the patch panel
to the racks.

\begin{figure}[h!]
  \begin{center}
    \scalebox{0.6}{\input{\pdftextimgpath/fig-calosignal_crate_cabling.pdftex_t}}
  \end{center}
  \caption{Calorimeter front-end crates cabling path.}
  \label{fig:calosignal:crates:3}
 
\end{figure}

%% end

% -*- mode: latex; -*-

\clearpage
\section{The Calorimeter signal cabling table and its usage}

\subsection{Source table}

Cabling the calorimeter readout system  consists in the association of
each  PMT  to a  Wavecatcher  channel.   An  unique cabling  table  is
provided to give an unambiguous  description of the signal cable paths
from  the calorimeter  front-end boards  to the  detector.  The  table
consists  in  an   associative  map  like  the  one   shown  on  table
\ref{tab:calosignal:map:1}.

\begin{table}[h]
\begin{center}
\begin{tabular}{|c|c|c|c|}
  \hline
  \textbf{Readout Channel}& \textbf{External signal cable} & \textbf{Internal signal cable} & \textbf{Optical Module} \\
  \hline
  \hline
  \texttt{H:0.0.0}   & \texttt{L:11.0}   & \texttt{A:11.0}   & \texttt{M:0.0.0}   \\
  \hline
  \texttt{H:0.0.1}   & \texttt{L:11.20}  & \texttt{A:11.20}  & \texttt{M:0.0.1}   \\
  \hline
  \texttt{H:0.0.2}   & \texttt{L:12.0}   & \texttt{A:12.0}   & \texttt{M:0.0.2}   \\
  \hline
  \vdots             & \vdots            &  \vdots           & \vdots             \\  
  \hline
\end{tabular}
\end{center}
\caption{Example of CaloSignal cabling table}
\label{tab:calosignal:map:1}
\end{table}

\par\noindent The table is provided in the form of a CSV\footnote{CSV:
  coma separated value} file.  The file uses the following format:

\begin{itemize}
\item The file contains only ASCII characters.
\item Blank lines are ignored.
\item Lines starting with the hashtag character \fbox{\texttt{\#}} are
  ignored, enabling to write some comments.
\item  There  is  only  one Wavecatcher  front-end  board  channel/PMT
  association per line.
\item Each  line has four  columns separated by  the \emph{semi-colon}
  character \fbox{\texttt{;}}.
\item The first column contains the label of the Wavecatcher front-end
  board readout channel.
\item The second column contains the label of the external signal cable.
\item The third column contains the label of the internal signal cable.
\item  The  fourth column  contains  the  label  of the  PMT  (optical
  module).
\end{itemize}

\par\noindent The  CaloSignal cabling map  file is used as  the unique
source of information for different purposes:
\begin{itemize}
\item generation of labels to be stuck on external and internal cables
  or harnesses;
\item  generation of  printable tables  for  people in  charge of  the
  calorimeter readout cabling at LSM,
\item input for dedicated software  modeling tools used by the Control
  and Monitoring System (CMS), the simulation\dots
\end{itemize}


\subsection{CaloSignal cabling sheets}


The  SNCabling  package  provides  a Python  script  to  automatically
generate, from the CaloSignal cabling  table, a printable PDF document
with cabling  tables corresponding  to each part  of the  detector and
each front-end crate.

\subsection{Labels}

The  SNCabling  package  provides  a Python  script  to  automatically
generate, from the  CaloSignal cabling table, the lists of  all labels for all
signal  harnesses  and  cables.   The  labels must  be  stuck  on  the
terminations of all  harnesses and cables to help the  cabling team to
identifiy   the    proper   connections   between    front-end   board
channels/external  cables/patch   panel/internal  cables/PMT.   Figure
\ref{fig:calosignal:principle:1} shows  where various kinds  of labels
are supposed to be stuck on signal harnesses and cables.

\begin{itemize}
\item Each label  stuck on the end of an  internal signal cable on
  the PMT  side identifies  not only  the signal cable  itself but  also the
  optical module/PMT it is connected to. Example:
  \begin{center}
    \fbox{\texttt{A:9:1 -> M:1.4.0}}
  \end{center}
\item Each label  stuck on the end of an  external signal cable on
  the front-end crate  side identifies  not only  the signal cable  itself but  also the
  Wavecatcher readout channel it is connected to.Example:
  \begin{center}
    \fbox{\texttt{L:9:1 -> H:1.1.0}}
  \end{center}
\end{itemize}

%% end


\end{document}
