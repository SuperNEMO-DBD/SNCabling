\documentclass[12pt,a4paper]{article}

\usepackage{a4wide}
\usepackage{verbatim}
\usepackage[T1]{fontenc}
\usepackage{ucs}
\usepackage[utf8x]{inputenc}
\usepackage[dvips]{color}
\usepackage{graphicx}
\usepackage{epic}
\usepackage{eepic}
\usepackage{eepicemu}
\usepackage{array}
\usepackage{moreverb}
\usepackage{fancyvrb}
\usepackage{url}
\usepackage{eurosym}
\usepackage{amsmath}
\usepackage{amssymb}
\usepackage{multicol}

\newcommand{\imgpath}{./images}
\newcommand{\samplespath}{./samples}
\newcommand{\pdfteximgpath}{./pdftex}
\newcommand{\pdftextimgpath}{./pdftex_t}

\title{SuperNEMO Demonstrator\\
  Calorimeter High Voltage System\\
  Cabling scheme and cable labels\\
  version 0.2}
\author{M.Bongrand, Y.Lemi\`ere, F.Mauger}
\date{November 1st, 2018}

%%%%%%%%%%%%%%%%
\begin{document}

\maketitle

\begin{abstract}
  \noindent This document presents the cable labelling convention used
  for  the SuperNEMO  Demonstrator's Calorimeter  High Voltage  System
  (CaloHV).  We reuse  here some informations available  from a couple
  of documents prepared  by Cedric and Mathieu,  with some adaptations
  and addons.

  \vskip 10pt
  \noindent This document and all associated tools
  are hosted at:
  \vskip 5pt
  \url{https://gitlab.in2p3.fr/SuperNEMO-DBD/SNCabling}.
  
\end{abstract}

\tableofcontents
\vfill

\clearpage
\section{Principle}

The SuperNEMO Demonstrator's Calorimeter  High Voltage System (CaloHV)
uses two CAEN High  Voltage crates which host a total  of 24 boards to
distribute HV to 712 PMTs.

Each HV board manages up to 32 channels and is connected to a specific
set of  PMTs through  a pair  of harnesses.   A first  harness, called
\emph{external harness}, links the board  to a single connector on the
patch panel (external side). From this connector on the patch panel, a
new harness, namely the \emph{internal  harness}, routes HV channel to
the PMTs.  The end of the  \emph{internal harness} is designed in such
a way cables can be routed  indivudually to their associated PMTs.  Of
course, a given harness handles a  set of PMTs that are geographically
close to each other, in order to optimize the length of the cables.

In the present  scheme, it has been decided to  identify individual HV
distribution cables using  the pin identifiers on  the connectors they
are associated  to.  Ideally, the  pin number  on the HV  board output
connector thus identifies the cable  linked to it.  This identifier is
propagated to the pin number on the patch panel connectors then to the
end  of the  cable linked  to the  PMT. This  enables to  build rather
simple and comprehensible cabling tables.

\noindent\par  Figure  \ref{fig:calohv:principle:1}  shows  the  basic
diagram of  the CaloHV  system. A dedicated  labelling system  will be
used to ease the cabling operations.

\begin{figure}[h!]
  \begin{center}
    \scalebox{0.75}{\input{\pdftextimgpath/fig-calohv-1.pdftex_t}}
  \end{center}
  \caption{Principle of  the HV distribution  to the PMTs  using the
    CAEN HV power supplies.}
  \label{fig:calohv:principle:1}
\end{figure}

\clearpage

\section{Addressing objects}

\subsection{Format of a CaloHV label}

Each label to be used for CaloHV cabling will use the following format:

\begin{center}
  \fbox{\texttt{X:}$id_1$.$id_2$\dots$id_n$}
\end{center}
\noindent  where \texttt{X}  is a  single letter  which describes  the
category  of the  labelled  object,  and the  $id_1$.$id_2$\dots$id_n$
sequence is the unique address of the object within its category.  The
$id_x$ tokens are positive integers (possibly zero).  The \emph{colon}
character is  used to separate  the category letter from  the address.
The  sequence  of  identifiers  in  the  address  uses  the  \emph{dot}
character as a separator.


\subsection{HV crates, boards, channel, harnesses and cables}

Each CAEN HV power supply crate belonging to the CaloHV system is installed
in the rack number 2 on the electronics platform.
A HV crate is identified with an unique ID, namely
a number ranging from \texttt{0} to \texttt{1}.  We propose to label a
given HV crate with the following scheme:
\begin{center}
   \fbox{\texttt{C:$crate$}}
 \end{center}
where \texttt{$crate$} is the number of the crate (positive integer).
\vskip     10pt    \par\noindent     Examples:    \fbox{\texttt{C:0}},
\fbox{\texttt{C:1}}.  \par Conventionally, crate \texttt{0} manages HV
for PMTs on the \emph{Italy} side  and crate \texttt{1} manages HV for
PMTs on the \emph{France} side.

\vskip 10pt A HV crate contains up to 16 HV 32-channel boards but only 12
will be  used.  A  HV board  inherits the  number of  the crate  it is
plugged into  and is addressed through  its slot number.  We give a given HV board
a label with the following scheme:
\begin{center}
   \fbox{\texttt{B:$crate$.$board$}}
 \end{center}
where \texttt{$crate$} is the number of the crate and \texttt{$board$}
is  the  number  of  the  board  (slot)  ranging  from  \texttt{0}  to
\texttt{15}.
\vskip 10pt
\par\noindent Examples: \fbox{\texttt{B:0.0}},  \dots
\fbox{\texttt{B:1.11}}.

\vskip 10pt
\par\noindent Up  to 32 HV channels are addressed within a HV board.
 We propose to label a given HV channel with the following scheme:
\begin{center}
   \fbox{\texttt{H:$crate$.$board$.$channel$}}
 \end{center}
where \texttt{$crate$} is the number of the crate,
  \texttt{$board$} is the number of the crate
and  \texttt{$channel$} is the number of the channel
ranging from
\texttt{0} to \texttt{31}.
\par\noindent Examples: \fbox{\texttt{H:0.0.0}}, \dots
\fbox{\texttt{H:1.11.31}}

\vskip 10pt
A HV channel is automatically associated to a specific pin number of
the output connector of the CAEN HV board. 

\vskip 10pt A external HV  harness  connecting a given HV board (Radial
connector) to the patch panel (Redel  Male S connector) uses an unique
ID  ranging from  \texttt{0} to  \texttt{23}.  We give to each
external HV harness a label with the following scheme:
\begin{center}
   \fbox{\texttt{E:$harness$}}
 \end{center}
where \texttt{$harness$} is the number of the external harness.
\par\noindent Examples: \fbox{\texttt{E:0}}, \dots
\fbox{\texttt{E:23}}

\vskip 10pt Compared with the  labelling scheme proposed by Mathieu in
original  documents,  it   has  been  decided  not   to  introduce  an
intermediate cable  identifier depending on  the location of  the PMTs
(main walls,  top row in  main walls,  X-walls, gamma veto rows).  A unique
scheme   is  used in place,   based   on  an   already  existing   informations,
independently of  the geometry.  Individual cables  within an external
harness are identified through the  pin numbers they are associated to
on  the   HV  board  output  connector.    This  pin/cable  identifier
propagates up  to the  patch panel  and beyond  to the  internal cable
terminations.  There is no need  to label internal cables because they
are   confined  within   their  harness   and  thus   never  addressed
individually during cabling operations.


\subsection{Optical modules}


The  identification scheme  of the  optical  modules and their PMTs
is  based on  the
addressing scheme defined in the geometry model and implemented in the
simulation     and     data    analysis     software\footnote{Falaise:
  \url{https://gitub.com/SuperNEMO-DBD/Falaise}}.     There   are    4
categories  of  optical  modules  and  thus  of  scintillator  blocks,
depending on their location in the experimental setup:

\begin{itemize}
  
\item Main wall block (Falaise: geometry category \texttt{"calorimeter\_block"}
  and  type  \texttt{1302}):

  \par  OMs are  addressed  through  their
  \emph{side}   number from 0 (Italy)     to  1 (France),
  \emph{column} number from 0 (Edelweiss) to 19 (Tunnel) and
  \emph{row}    number from 0 (bottom)    to 12 (top).

  \par We  propose to label such a block with the following scheme:
  \begin{center}
    \fbox{\texttt{M:$side$.$column$.$row$}}
  \end{center}
  \vskip 10pt
  \par\noindent Examples: \fbox{\texttt{M:0.0.0}}, 
  \fbox{\texttt{M:0.19.12}}, \fbox{\texttt{M:1.0.0}}, \fbox{\texttt{M:1.19.12}}.
  
\item X-wall block (Falaise: geometry category \texttt{"xcalo\_block"} and type
  \texttt{1232}):
  
  \par OMs  are addressed  through their
  \emph{side}   number from 0 (Italy)     to  1 (France),
  \emph{wall}   number from 0 (Edelweiss) to  1 (tunnel),
  \emph{column} number from 0 (source)    to  1 (calorimeter) and
  \emph{row}    number from 0 (bottom)    to 15 (top).
  
  \par  We propose  to
  label such a block with the following scheme:
  \begin{center}
    \fbox{\texttt{X:$side$.$wall$.$column$.$row$}}
  \end{center}
  \vskip 10pt
  \par\noindent Examples: \fbox{\texttt{X:0.1.1.15}}, \fbox{\texttt{X:1.0.0.8}}
  
\item Gamma veto block  (Falaise: geometry category \texttt{"gveto\_block"} and
  type   \texttt{1252}):
  
  \par   OMs  are   addressed  through   their
  \emph{side}   number from 0 (Italy)     to  1 (France),
  \emph{wall}   number from 0 (bottom)    to  1 (top)  and
  \emph{column} number from 0 (Edelweiss) to 15 (tunnel).
  
  \par We propose to label  such a block
  with the following scheme:
  \begin{center}
    \fbox{\texttt{G:$side$.$wall$.$column$}}
  \end{center}
  \vskip 10pt
  
  \par\noindent Examples: \fbox{\texttt{G:0.1.0}}, \fbox{\texttt{G:1.0.8}}
  
\item  Block for  reference  optical module:
  
  \par  OMs are  addressed  through their \emph{ref}  number.
  
  \par We propose to  label such a  block with the following scheme:
  \begin{center}
    \fbox{\texttt{R:$ref$}}
  \end{center}

\end{itemize}



\clearpage
\section{The CaloHV crates}

Figures  \ref{fig:calohv:crates:0} and  \ref{fig:calohv:crates:1} show
 the repartition  of HV boards  respectively within the CAEN  HV power
 supply crates 0 (Italy) and 1 (France).

\begin{figure}[h!]
  \begin{center}
    \scalebox{0.6}{\input{\pdftextimgpath/fig-calohv-2-italy.pdftex_t}}
  \end{center}
  \caption{CaloHV crate 0 (Italy).}
  \label{fig:calohv:crates:0}
\end{figure}

\begin{figure}[h!]
  \begin{center}
    \scalebox{0.6}{\input{\pdftextimgpath/fig-calohv-2-france.pdftex_t}}
  \end{center}
  \caption{CaloHV crate 1 (France).}
  \label{fig:calohv:crates:1}
\end{figure}

%% end

% -*- mode: latex; -*-

\section{The CaloHV cabling table and its usage}

\subsection{Source table}

Cabling the CaloHV consists in the association of each OM's PMT to one
HV  cable.  An  unique  cabling  table must  be  provided  to give  an
unambiguous description of the cables' path  from the HV boards to the
detector.  The table consists in an associative map like the one shown
on table \ref{tab:calohv:map:1}.

\begin{table}[h]
\begin{center}
\begin{tabular}{|c|c|c|c|}
  \hline
  \textbf{HV channel}& \textbf{External harness} & \textbf{Internal cable} & \textbf{Optical Module} \\
  \hline
  %% \hline
  %% \multicolumn{4}{|l|}{\textbf{HV board} \texttt{B:0.1}}  \\
  \hline
  \texttt{H:0.0.23}  & \texttt{E:12} & \texttt{A:12.3} & \texttt{M:0.0.0} \\
  \hline
  \texttt{H:0.0.22}  & \texttt{E:12} & \texttt{A:12.18} & \texttt{M:0.0.1} \\
  \hline
  \texttt{H:0.0.6}  & \texttt{E:12} & \texttt{A:12.34} & \texttt{M:0.0.2} \\
  \hline
  \texttt{H:0.2.23}  & \texttt{E:14} & \texttt{A:14.3} & \texttt{M:0.0.3} \\
  \hline
  \vdots          & & & \vdots             \\  
  \hline
  \hline
\end{tabular}
\end{center}
\caption{Example of CaloHV cabling table}
\label{tab:calohv:map:1}
\end{table}

\par\noindent  The   table  will  be   provided  in  the  form   of  a
CSV\footnote{CSV: coma  separated value} file.  The file must  use the
following format:

\begin{itemize}
\item The file contains only ASCII characters.
\item Blank lines are ignored.
\item Lines starting with the hashtag character \fbox{\texttt{\#}} are
  ignored, enabling to write some comments.
\item There is only one HV channel/PMT association per line.
\item Each  line has  four columns  separated by  the \emph{semi-colon}
  character \fbox{\texttt{;}}.
\item The first column contains the label of the HV channel.
\item The second column contains the label of the external bundle.
\item The third column contains the label of the internal cable.
\item The fourth column contains the label of the PMT (optical module).
\end{itemize}

\par\noindent The  CaloHV cabling  map file  will be  used as  the unique
source of information for different purposes:
\begin{itemize}
\item generation of labels to be stuck on cables, harnesses, patch panel;
\item generation of  printable tables for people in charge  of the CaloHV
  cabling at LSM,
\item input for dedicated software  modeling tools used by the
  Control and Monitoring System (CMS), the simulation\dots 
\end{itemize}


\subsection{CaloHV cabling sheets}


From  the CaloHV  cabling  table, a  Python script  will  be provided  to
generate a printable PDF document with cabling tables corresponding to
each  part of  the  detector.

%% A typical  output  is  shown on  figure
%% \ref{fig:lis:sheet:1}.


%% \begin{figure}[h!]
%%   \begin{center}
%%     \scalebox{0.58}{\begin{tabular}{|r||c|c|c|c|c|c|c|c|c|c|c|c|c|c|c|c|c|c|c|c||l|}
\hline
 & \textcolor{blue}{\small 19} & \textcolor{blue}{\small 18} & \textcolor{blue}{\small 17} & \textcolor{blue}{\small 16} & \textcolor{blue}{\small 15} & \textcolor{blue}{\small 14} & \textcolor{blue}{\small 13} & \textcolor{blue}{\small 12} & \textcolor{blue}{\small 11} & \textcolor{blue}{\small 10} & \textcolor{blue}{\small 9} & \textcolor{blue}{\small 8} & \textcolor{blue}{\small 7} & \textcolor{blue}{\small 6} & \textcolor{blue}{\small 5} & \textcolor{blue}{\small 4} & \textcolor{blue}{\small 3} & \textcolor{blue}{\small 2} & \textcolor{blue}{\small 1} & \textcolor{blue}{\small 0} & \\
\hline
\hline
\textcolor{blue}{\small 12} & P:?.?  &  P:?.?  &  P:?.?  &  P:?.?  &  P:?.?  &  P:?.?  &  P:?.?  &  P:?.?  &  P:?.?  &  P:?.?  &  P:?.?  &  P:?.?  &  P:?.?  &  P:?.?  &  P:?.?  &  P:?.?  &  P:?.?  &  P:?.?  &  P:?.?  &  P:?.? & \textcolor{blue}{\small 12}\\
 & S:?.?  &  S:?.?  &  S:?.?  &  S:?.?  &  S:?.?  &  S:?.?  &  S:?.?  &  S:?.?  &  S:?.?  &  S:?.?  &  S:?.?  &  S:?.?  &  S:?.?  &  S:?.?  &  S:?.?  &  S:?.?  &  S:?.?  &  S:?.?  &  S:?.?  &  S:?.? & \\
\hline
\textcolor{blue}{\small 11} & P:?.?  &  P:?.?  &  P:?.?  &  P:?.?  &  P:?.?  &  P:?.?  &  P:?.?  &  P:?.?  &  P:?.?  &  P:?.?  &  P:?.?  &  P:?.?  &  P:?.?  &  P:?.?  &  P:?.?  &  P:?.?  &  P:?.?  &  P:?.?  &  P:?.?  &  P:?.? & \textcolor{blue}{\small 11}\\
 & S:?.?  &  S:?.?  &  S:?.?  &  S:?.?  &  S:?.?  &  S:?.?  &  S:?.?  &  S:?.?  &  S:?.?  &  S:?.?  &  S:?.?  &  S:?.?  &  S:?.?  &  S:?.?  &  S:?.?  &  S:?.?  &  S:?.?  &  S:?.?  &  S:?.?  &  S:?.? & \\
\hline
\textcolor{blue}{\small 10} & P:?.?  &  P:?.?  &  P:?.?  &  P:?.?  &  P:?.?  &  P:?.?  &  P:?.?  &  P:?.?  &  P:?.?  &  P:?.?  &  P:?.?  &  P:?.?  &  P:?.?  &  P:?.?  &  P:?.?  &  P:?.?  &  P:?.?  &  P:?.?  &  P:?.?  &  P:?.? & \textcolor{blue}{\small 10}\\
 & S:?.?  &  S:?.?  &  S:?.?  &  S:?.?  &  S:?.?  &  S:?.?  &  S:?.?  &  S:?.?  &  S:?.?  &  S:?.?  &  S:?.?  &  S:?.?  &  S:?.?  &  S:?.?  &  S:?.?  &  S:?.?  &  S:?.?  &  S:?.?  &  S:?.?  &  S:?.? & \\
\hline
\textcolor{blue}{\small 9} & P:?.?  &  P:?.?  &  P:?.?  &  P:?.?  &  P:?.?  &  P:?.?  &  P:?.?  &  P:?.?  &  P:?.?  &  P:?.?  &  P:?.?  &  P:?.?  &  P:?.?  &  P:?.?  &  P:?.?  &  P:?.?  &  P:?.?  &  P:?.?  &  P:?.?  &  P:?.? & \textcolor{blue}{\small 9}\\
 & S:?.?  &  S:?.?  &  S:?.?  &  S:?.?  &  S:?.?  &  S:?.?  &  S:?.?  &  S:?.?  &  S:?.?  &  S:?.?  &  S:?.?  &  S:?.?  &  S:?.?  &  S:?.?  &  S:?.?  &  S:?.?  &  S:?.?  &  S:?.?  &  S:?.?  &  S:?.? & \\
\hline
\textcolor{blue}{\small 8} & P:?.?  &  P:?.?  &  P:?.?  &  P:?.?  &  P:?.?  &  P:?.?  &  P:?.?  &  P:?.?  &  P:?.?  &  P:?.?  &  P:?.?  &  P:?.?  &  P:?.?  &  P:?.?  &  P:?.?  &  P:?.?  &  P:?.?  &  P:?.?  &  P:?.?  &  P:?.? & \textcolor{blue}{\small 8}\\
 & S:?.?  &  S:?.?  &  S:?.?  &  S:?.?  &  S:?.?  &  S:?.?  &  S:?.?  &  S:?.?  &  S:?.?  &  S:?.?  &  S:?.?  &  S:?.?  &  S:?.?  &  S:?.?  &  S:?.?  &  S:?.?  &  S:?.?  &  S:?.?  &  S:?.?  &  S:?.? & \\
\hline
\textcolor{blue}{\small 7} & P:?.?  &  P:?.?  &  P:?.?  &  P:?.?  &  P:?.?  &  P:?.?  &  P:?.?  &  P:?.?  &  P:?.?  &  P:?.?  &  P:?.?  &  P:?.?  &  P:?.?  &  P:?.?  &  P:?.?  &  P:?.?  &  P:?.?  &  P:?.?  &  P:?.?  &  P:?.? & \textcolor{blue}{\small 7}\\
 & S:?.?  &  S:?.?  &  S:?.?  &  S:?.?  &  S:?.?  &  S:?.?  &  S:?.?  &  S:?.?  &  S:?.?  &  S:?.?  &  S:?.?  &  S:?.?  &  S:?.?  &  S:?.?  &  S:?.?  &  S:?.?  &  S:?.?  &  S:?.?  &  S:?.?  &  S:?.? & \\
\hline
\textcolor{blue}{\small 6} & P:?.?  &  P:?.?  &  P:?.?  &  P:?.?  &  P:?.?  &  P:?.?  &  P:?.?  &  P:?.?  &  P:?.?  &  P:?.?  &  P:?.?  &  P:?.?  &  P:?.?  &  P:?.?  &  P:?.?  &  P:?.?  &  P:?.?  &  P:?.?  &  P:?.?  &  P:?.? & \textcolor{blue}{\small 6}\\
 & S:?.?  &  S:?.?  &  S:?.?  &  S:?.?  &  S:?.?  &  S:?.?  &  S:?.?  &  S:?.?  &  S:?.?  &  S:?.?  &  S:?.?  &  S:?.?  &  S:?.?  &  S:?.?  &  S:?.?  &  S:?.?  &  S:?.?  &  S:?.?  &  S:?.?  &  S:?.? & \\
\hline
\textcolor{blue}{\small 5} & P:?.?  &  P:?.?  &  P:?.?  &  P:?.?  &  P:?.?  &  P:?.?  &  P:?.?  &  P:?.?  &  P:?.?  &  P:?.?  &  P:?.?  &  P:?.?  &  P:?.?  &  P:?.?  &  P:?.?  &  P:?.?  &  P:?.?  &  P:?.?  &  P:?.?  &  P:?.? & \textcolor{blue}{\small 5}\\
 & S:?.?  &  S:?.?  &  S:?.?  &  S:?.?  &  S:?.?  &  S:?.?  &  S:?.?  &  S:?.?  &  S:?.?  &  S:?.?  &  S:?.?  &  S:?.?  &  S:?.?  &  S:?.?  &  S:?.?  &  S:?.?  &  S:?.?  &  S:?.?  &  S:?.?  &  S:?.? & \\
\hline
\textcolor{blue}{\small 4} & P:?.?  &  P:?.?  &  P:?.?  &  P:?.?  &  P:?.?  &  P:?.?  &  P:?.?  &  P:?.?  &  P:?.?  &  P:?.?  &  P:?.?  &  P:?.?  &  P:?.?  &  P:?.?  &  P:?.?  &  P:?.?  &  P:?.?  &  P:?.?  &  P:?.?  &  P:?.? & \textcolor{blue}{\small 4}\\
 & S:?.?  &  S:?.?  &  S:?.?  &  S:?.?  &  S:?.?  &  S:?.?  &  S:?.?  &  S:?.?  &  S:?.?  &  S:?.?  &  S:?.?  &  S:?.?  &  S:?.?  &  S:?.?  &  S:?.?  &  S:?.?  &  S:?.?  &  S:?.?  &  S:?.?  &  S:?.? & \\
\hline
\textcolor{blue}{\small 3} & P:?.?  &  P:?.?  &  P:?.?  &  P:?.?  &  P:?.?  &  P:?.?  &  P:?.?  &  P:?.?  &  P:?.?  &  P:?.?  &  P:?.?  &  P:?.?  &  P:?.?  &  P:?.?  &  P:?.?  &  P:?.?  &  P:?.?  &  P:?.?  &  P:?.?  &  P:?.? & \textcolor{blue}{\small 3}\\
 & S:?.?  &  S:?.?  &  S:?.?  &  S:?.?  &  S:?.?  &  S:?.?  &  S:?.?  &  S:?.?  &  S:?.?  &  S:?.?  &  S:?.?  &  S:?.?  &  S:?.?  &  S:?.?  &  S:?.?  &  S:?.?  &  S:?.?  &  S:?.?  &  S:?.?  &  S:?.? & \\
\hline
\textcolor{blue}{\small 2} & P:?.?  &  P:?.?  &  P:?.?  &  P:?.?  &  P:?.?  &  P:?.?  &  P:?.?  &  P:?.?  &  P:?.?  &  P:?.?  &  P:?.?  &  P:?.?  &  P:?.?  &  P:?.?  &  P:?.?  &  P:?.?  &  P:?.?  &  P:?.?  &  P:?.?  &  P:?.? & \textcolor{blue}{\small 2}\\
 & S:?.?  &  S:?.?  &  S:?.?  &  S:?.?  &  S:?.?  &  S:?.?  &  S:?.?  &  S:?.?  &  S:?.?  &  S:?.?  &  S:?.?  &  S:?.?  &  S:?.?  &  S:?.?  &  S:?.?  &  S:?.?  &  S:?.?  &  S:?.?  &  S:?.?  &  S:?.? & \\
\hline
\textcolor{blue}{\small 1} & P:?.?  &  P:?.?  &  P:?.?  &  P:?.?  &  P:?.?  &  P:?.?  &  P:?.?  &  P:?.?  &  P:?.?  &  P:?.?  &  P:?.?  &  P:?.?  &  P:?.?  &  P:?.?  &  P:?.?  &  P:?.?  &  P:?.?  &  P:?.?  &  P:?.?  &  P:?.? & \textcolor{blue}{\small 1}\\
 & S:?.?  &  S:?.?  &  S:?.?  &  S:?.?  &  S:?.?  &  S:?.?  &  S:?.?  &  S:?.?  &  S:?.?  &  S:?.?  &  S:?.?  &  S:?.?  &  S:?.?  &  S:?.?  &  S:?.?  &  S:?.?  &  S:?.?  &  S:?.?  &  S:?.?  &  S:?.? & \\
\hline
\textcolor{blue}{\small 0} & P:?.?  &  P:?.?  &  P:?.?  &  P:?.?  &  P:?.?  &  P:?.?  &  P:?.?  &  P:?.?  &  P:?.?  &  P:?.?  &  P:?.?  &  P:?.?  &  P:?.?  &  P:?.?  &  P:?.?  &  P:?.?  &  P:?.?  &  P:?.?  &  P:?.?  &  P:?.? & \textcolor{blue}{\small 0}\\
 & S:?.?  &  S:?.?  &  S:?.?  &  S:?.?  &  S:?.?  &  S:?.?  &  S:?.?  &  S:?.?  &  S:?.?  &  S:?.?  &  S:?.?  &  S:?.?  &  S:?.?  &  S:?.?  &  S:?.?  &  S:?.?  &  S:?.?  &  S:?.?  &  S:?.?  &  S:?.? & \\
\hline
\hline
 & \textcolor{blue}{\small 19} & \textcolor{blue}{\small 18} & \textcolor{blue}{\small 17} & \textcolor{blue}{\small 16} & \textcolor{blue}{\small 15} & \textcolor{blue}{\small 14} & \textcolor{blue}{\small 13} & \textcolor{blue}{\small 12} & \textcolor{blue}{\small 11} & \textcolor{blue}{\small 10} & \textcolor{blue}{\small 9} & \textcolor{blue}{\small 8} & \textcolor{blue}{\small 7} & \textcolor{blue}{\small 6} & \textcolor{blue}{\small 5} & \textcolor{blue}{\small 4} & \textcolor{blue}{\small 3} & \textcolor{blue}{\small 2} & \textcolor{blue}{\small 1} & \textcolor{blue}{\small 0} & \\
\hline
\end{tabular}
}
%%   \end{center}
%%   \caption{Expected  printable   LIS  cabling  table  for   the  Italy
%%     calorimeter  main  wall  (front  view).  Question  marks  will  be
%%     replaced by the real fibers' identifiers when the final table will
%%     be  available. Similar  tables will  be provided  for X-walls  and
%%     gamma veto lines.}
%%   \label{fig:lis:sheet:1}
%% \end{figure}


\subsection{LIS labels}

Practically, we  plan to generate  labels for all harnesses and cables
to  help  the  cabling  team  to  identifiy  the  proper  board/patch panel
and internal cable/PMT connections.
The label  will be prepared and stuck near
the terminations of the harnesses and cables.

%% \noindent Example below displays the  full text expected to be printed
%% on a fiber label:
%% \begin{center}
%% \fbox{\texttt{P:2.34} $\rightarrow$ \texttt{M:0.2.5}}
%% \end{center}

%% \par\noindent Such  informations are of course  redundant with respect
%% to  the  printable   plain  table  shown  in   the  previous  section.
%% The  fiber labels could  possibly be removed before  installing the
%% coil.

%% \par\noindent We  consider to write  a Python script  to automatically
%% generate all labels (up to 1400).


\end{document}
