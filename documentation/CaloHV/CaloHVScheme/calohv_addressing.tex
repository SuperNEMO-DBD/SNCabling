
\section{Addressing objects}

\subsection{Format of a CaloHV label}

Each label used for CaloHV cabling uses the following format:

\begin{center}
  \fbox{\texttt{X:}$id_1$.$id_2$\dots$id_n$}
\end{center}
\noindent  where \texttt{X}  is a single letter  which indentifies  the
category  of the  labelled  object,  and the  $id_1$.$id_2$\dots$id_n$
sequence is the unique address of the object within its category.  The
$id_x$ tokens are positive integers (possibly zero).  The \emph{colon}
character is  used to separate  the category letter from  the address.
The  sequence  of  identifiers  in  the  address  uses  the  \emph{dot}
character as a separator.

\vskip 10pt Examples: \texttt{H:1.3.4}, \texttt{A:6.3}.


\subsection{HV crates, boards and channels}

Each CAEN  HV power  supply crate  belonging to  the CaloHV  system is
installed in  the rack  number 2  on the  electronics platform.   A HV
crate is  identified with an unique  ID, namely a number  ranging from
\texttt{0}  to \texttt{1}.   A given  HV  crate is  labelled with  the
following scheme:
\begin{center}
   \fbox{\texttt{C:$crate$}}
 \end{center}
where \texttt{$crate$} is the number of the crate (positive integer).
\vskip     10pt    \par\noindent     Examples:    \fbox{\texttt{C:0}},
\fbox{\texttt{C:1}}.  \par Conventionally, crate \texttt{0} manages HV
for PMTs on the \emph{Italy} side  and crate \texttt{1} manages HV for
PMTs on the \emph{France} side  (including main wall, X-wall and gamma
veto optical modules).

\vskip 10pt A HV crate contains up to 16 HV 32-channel boards but only
12 will  be used  per crate.  A  HV board inherits  the number  of the
crate it is plugged into and  is addressed through its slot number.  A
given HV board is identified by a label with the following scheme:
\begin{center}
   \fbox{\texttt{B:$crate$.$board$}}
 \end{center}
where \texttt{$crate$} is the number of the crate and \texttt{$board$}
is  the  number  of  the  board  (slot)
%% in the following list: \texttt{0,1,3,4,6,7,}
ranging from \texttt{0} to \texttt{11}.
\texttt{15}.
\vskip 10pt
\par\noindent Examples: \fbox{\texttt{B:0.0}},  \dots
\fbox{\texttt{B:1.11}}.

\vskip 10pt \par\noindent Up to 32  HV channels are addressed within a
HV  board.  A  given HV  channel  is identified  by a  label with  the
following scheme:
\begin{center}
   \fbox{\texttt{H:$crate$.$board$.$channel$}}
 \end{center}
where \texttt{$crate$} is the number of the crate, \texttt{$board$} is
the number  of the crate and  \texttt{$channel$} is the number  of the
channel  ranging   from  \texttt{0}  to   \texttt{31}.
\vskip 10pt\par\noindent
Examples: \fbox{\texttt{H:0.0.0}}, \dots \fbox{\texttt{H:1.11.31}}

\vskip 10pt A HV channel is automatically associated to a specific pin
number  of  the  output  connector   of  the  CAEN  HV  board.   Table
\ref{calohv:channel_pin_map}  shows the  associative  map between  the
CAEN HV board channel numbers and Radial connector pins.

\begin{table}[h!]
\begin{center}
  \begin{tabular}{|c|c|}
    \hline
    HV channel & Radial connector pin \\
    \hline
    \hline
    0 & 45  \\
    1 & 46  \\
    2 & 47  \\
    3 & 48  \\
    4 & 49  \\
    5 & 50  \\
    6 & 34  \\
    7 & 35  \\
    8 & 36  \\
    9 & 37  \\
    10 & 38   \\
    11 & 39  \\
    12 & 24  \\
    13 & 25  \\
    14 & 26  \\
    15 & 27  \\
    16 & 28  \\
    17 & 29  \\
    18 & 14  \\
    19 & 15  \\
    20 & 16  \\
    21 & 17  \\
    22 & 18  \\
    23 & 3  \\
    24 & 4  \\
    25 & 5  \\
    26 & 6  \\
    27 & 7  \\
    28 & 8  \\
    29 & 19  \\
    30 & 20  \\
    31 & 30  \\
   \hline
    \end{tabular}
  \caption{Associative map between HV board channels and 
    Radial connector's pins}\label{calohv:channel_pin_map}
\end{center}
\end{table}


\subsection{Harnesses and cables}

\vskip 10pt An external HV harness connecting a given HV board (Radial
connector) to the patch panel (Redel  Male S connector) uses an unique
ID ranging from  \texttt{0} to \texttt{23}.  An external  HV harness is
identified by a label with the following scheme:
\begin{center}
   \fbox{\texttt{E:$harness$}}
 \end{center}
where \texttt{$harness$} is the number of the external harness.
\par\noindent Examples: \fbox{\texttt{E:0}}, \dots
\fbox{\texttt{E:23}}

\vskip 10pt An internal HV harness connecting a given external harness
on  the patch  panel (Redel  connectors) to  a PMT  uses an  unique ID
ranging from  \texttt{0} to  \texttt{23}.  An  internal HV  harness is
identified by a label with the following scheme:
\begin{center}
   \fbox{\texttt{I:$harness$}}
 \end{center}
where  \texttt{$harness$}  is  the  number of  the  internal  harness.
\par\noindent        Examples:       \fbox{\texttt{I:0}},        \dots
\fbox{\texttt{I:23}}

\vskip 10pt Compared with the  labelling scheme proposed by Mathieu in
preliminary  documents,  it  has  been decided  not  to  introduce  an
intermediate cable  identifier depending on  the location of  the PMTs
(main walls,  top row  in main  walls, X-walls,  gamma veto  rows).  A
unique  scheme   is  used   in  place,   based  on   already  existing
informations, independently of the geometry.  Individual cables within
an external  harness are identified  through the pin numbers  they are
associated to on  the CAEN HV board output  connector.  This pin/cable
identifier propagates up to the patch panel and beyond to the internal
cable  terminations.  There  is no  need to  label external  HV cables
because  they  are  confined  within  their  harness  and  thus  never
addressed individually  during cabling operations.  Due  to the layout
of the CAEN HV board connectors, the pin number takes its value in the
list shown in table \ref{calohv:channel_pin_map}.

\vskip 10pt An  internal HV cable is identified by its label with the
following scheme:
\begin{center}
   \fbox{\texttt{A:$harness$.$pin$}}
 \end{center}
where \texttt{$harness$}  is the  number of  the internal  harness the
cable belongs to and \texttt{$pin$} is the number of the pin the cable
is  associated  to.   \par\noindent  Examples:  \fbox{\texttt{A:0.3}},
\dots \fbox{\texttt{A:0.50}}

\subsection{Optical modules}


The  identification scheme  of the  optical  modules and their PMTs
is  based on  the
addressing scheme defined in the geometry model and implemented in the
simulation     and     data    analysis     software\footnote{Falaise:
  \url{https://gitub.com/SuperNEMO-DBD/Falaise}}.     There   are    4
categories  of  optical  modules  and  thus  of  scintillator  blocks,
depending on their location in the experimental setup:

\begin{itemize}
  
\item Main wall block (Falaise: geometry category \texttt{"calorimeter\_block"}
  and  type  \texttt{1302}):

  \par  OMs are  addressed  through  their
  \emph{side}   number from 0 (Italy)     to  1 (France),
  \emph{column} number from 0 (Edelweiss) to 19 (Tunnel) and
  \emph{row}    number from 0 (bottom)    to 12 (top).

  \par We  propose to label such a block with the following scheme:
  \begin{center}
    \fbox{\texttt{M:$side$.$column$.$row$}}
  \end{center}
  \vskip 10pt
  \par\noindent Examples: \fbox{\texttt{M:0.0.0}}, 
  \fbox{\texttt{M:0.19.12}}, \fbox{\texttt{M:1.0.0}}, \fbox{\texttt{M:1.19.12}}.
  
\item X-wall block (Falaise: geometry category \texttt{"xcalo\_block"} and type
  \texttt{1232}):
  
  \par OMs  are addressed  through their
  \emph{side}   number from 0 (Italy)     to  1 (France),
  \emph{wall}   number from 0 (Edelweiss) to  1 (tunnel),
  \emph{column} number from 0 (source)    to  1 (calorimeter) and
  \emph{row}    number from 0 (bottom)    to 15 (top).
  
  \par  We propose  to
  label such a block with the following scheme:
  \begin{center}
    \fbox{\texttt{X:$side$.$wall$.$column$.$row$}}
  \end{center}
  \vskip 10pt
  \par\noindent Examples: \fbox{\texttt{X:0.1.1.15}}, \fbox{\texttt{X:1.0.0.8}}
  
\item Gamma veto block  (Falaise: geometry category \texttt{"gveto\_block"} and
  type   \texttt{1252}):
  
  \par   OMs  are   addressed  through   their
  \emph{side}   number from 0 (Italy)     to  1 (France),
  \emph{wall}   number from 0 (bottom)    to  1 (top)  and
  \emph{column} number from 0 (Edelweiss) to 15 (tunnel).
  
  \par We propose to label  such a block
  with the following scheme:
  \begin{center}
    \fbox{\texttt{G:$side$.$wall$.$column$}}
  \end{center}
  \vskip 10pt
  
  \par\noindent Examples: \fbox{\texttt{G:0.1.0}}, \fbox{\texttt{G:1.0.8}}
  
\item  Block for  reference  optical module:
  
  \par  OMs are  addressed  through their \emph{ref}  number.
  
  \par We propose to  label such a  block with the following scheme:
  \begin{center}
    \fbox{\texttt{R:$ref$}}
  \end{center}

\end{itemize}


\subsection{Summary}

\begin{center}
\begin{tabular}{|c|c|c|}
  \hline
  Category & Symbol & Depth \\
  \hline
  \hline
  \textbf{C}rate & \texttt{C} & 1 \\
  \hline
  \textbf{B}oard & \texttt{B} & 2 \\
  \hline
  C\textbf{h}annel & \texttt{H} & 3 \\
  \hline
  \textbf{E}xternal harness & \texttt{E} & 1 \\
  \hline
  \textbf{I}nternal harness & \texttt{I} & 1 \\
  \hline
  Internal c\textbf{A}ble & \texttt{A} & 2 \\
  \hline
  \textbf{M}ain wall PMT & \texttt{M} & 3 \\
  \hline
  \textbf{X}-wall PMT & \texttt{X} & 4 \\
  \hline
  \textbf{G}amma veto PMT & \texttt{G} & 3 \\
  \hline
\end{tabular}
\end{center}

%% end
