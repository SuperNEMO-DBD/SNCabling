% -*- mode: latex; -*-

\section{The CaloHV cabling table and its usage}

\subsection{Source table}

Cabling the CaloHV system consists in the association of each PMT to a
CAEN HV  board channel.  An unique  cabling table must be  provided to
give an unambiguous description of the  cable paths from the HV boards
to the  PMTs.  The table consists  in an associative map  like the one
shown on  table \ref{tab:calohv:map:1}.  This map contains  the needed
informations to ensure the addressing of all HV cables.

\begin{table}[h]
\begin{center}
\begin{tabular}{|c|c|c|c|}
  \hline
  \textbf{HV channel}& \textbf{External harness} & \textbf{Internal cable} & \textbf{Optical Module} \\
  \hline
  %% \hline
  %% \multicolumn{4}{|l|}{\textbf{HV board} \texttt{B:0.1}}  \\
  \hline
  \texttt{H:0.0.23}  & \texttt{E:12} & \texttt{A:12.3} & \texttt{M:0.0.0} \\
  \hline
  \texttt{H:0.0.22}  & \texttt{E:12} & \texttt{A:12.18} & \texttt{M:0.0.1} \\
  \hline
  \texttt{H:0.0.6}  & \texttt{E:12} & \texttt{A:12.34} & \texttt{M:0.0.2} \\
  \hline
  \texttt{H:0.2.23}  & \texttt{E:14} & \texttt{A:14.3} & \texttt{M:0.0.3} \\
  \hline
  \vdots          & \vdots  & \vdots   & \vdots             \\  
  \hline
\end{tabular}
\end{center}
\caption{Example of CaloHV cabling table}
\label{tab:calohv:map:1}
\end{table}

\par\noindent The table is provided in the form of a CSV\footnote{CSV:
  coma separated value} file.  The file must use the following format:

\begin{itemize}
\item The file contains only ASCII characters.
\item Blank lines are ignored.
\item Lines starting with the hashtag character \fbox{\texttt{\#}} are
  ignored, enabling to write some comments.
\item There is only one HV channel/PMT association per line.
\item Each  line has  four columns  separated by  the \emph{semi-colon}
  character \fbox{\texttt{;}}.
\item The first column contains the label of the CAEN HV board channel.
\item The second column contains the label of the external HV harness.
\item The third column contains the label of the internal HV cable.
\item The fourth column contains the label of the PMT (optical module).
\end{itemize}

\par\noindent The  CaloHV cabling map file  can be used as  the unique
source of information for different purposes:
\begin{itemize}
\item generation of labels to be stuck on internal HV cables, internal and external HV harnesses,
\item generation of  printable cabling tables for people in charge  of the calorimeter HV
  cabling at LSM,
\item input for dedicated software  modeling tools used by the
  Control and Monitoring System (CMS), the simulation\dots 
\end{itemize}


\subsection{CaloHV cabling sheets}


A  Python  script is  provided  to  automatically generate,  from  the
CaloHV cabling table, a printable PDF document with cabling tables
corresponding to each part of the detector.

%% A typical  output  is  shown on  figure
%% \ref{fig:lis:sheet:1}.


%% \begin{figure}[h!]
%%   \begin{center}
%%     \scalebox{0.58}{\begin{tabular}{|r||c|c|c|c|c|c|c|c|c|c|c|c|c|c|c|c|c|c|c|c||l|}
\hline
 & \textcolor{blue}{\small 19} & \textcolor{blue}{\small 18} & \textcolor{blue}{\small 17} & \textcolor{blue}{\small 16} & \textcolor{blue}{\small 15} & \textcolor{blue}{\small 14} & \textcolor{blue}{\small 13} & \textcolor{blue}{\small 12} & \textcolor{blue}{\small 11} & \textcolor{blue}{\small 10} & \textcolor{blue}{\small 9} & \textcolor{blue}{\small 8} & \textcolor{blue}{\small 7} & \textcolor{blue}{\small 6} & \textcolor{blue}{\small 5} & \textcolor{blue}{\small 4} & \textcolor{blue}{\small 3} & \textcolor{blue}{\small 2} & \textcolor{blue}{\small 1} & \textcolor{blue}{\small 0} & \\
\hline
\hline
\textcolor{blue}{\small 12} & P:?.?  &  P:?.?  &  P:?.?  &  P:?.?  &  P:?.?  &  P:?.?  &  P:?.?  &  P:?.?  &  P:?.?  &  P:?.?  &  P:?.?  &  P:?.?  &  P:?.?  &  P:?.?  &  P:?.?  &  P:?.?  &  P:?.?  &  P:?.?  &  P:?.?  &  P:?.? & \textcolor{blue}{\small 12}\\
 & S:?.?  &  S:?.?  &  S:?.?  &  S:?.?  &  S:?.?  &  S:?.?  &  S:?.?  &  S:?.?  &  S:?.?  &  S:?.?  &  S:?.?  &  S:?.?  &  S:?.?  &  S:?.?  &  S:?.?  &  S:?.?  &  S:?.?  &  S:?.?  &  S:?.?  &  S:?.? & \\
\hline
\textcolor{blue}{\small 11} & P:?.?  &  P:?.?  &  P:?.?  &  P:?.?  &  P:?.?  &  P:?.?  &  P:?.?  &  P:?.?  &  P:?.?  &  P:?.?  &  P:?.?  &  P:?.?  &  P:?.?  &  P:?.?  &  P:?.?  &  P:?.?  &  P:?.?  &  P:?.?  &  P:?.?  &  P:?.? & \textcolor{blue}{\small 11}\\
 & S:?.?  &  S:?.?  &  S:?.?  &  S:?.?  &  S:?.?  &  S:?.?  &  S:?.?  &  S:?.?  &  S:?.?  &  S:?.?  &  S:?.?  &  S:?.?  &  S:?.?  &  S:?.?  &  S:?.?  &  S:?.?  &  S:?.?  &  S:?.?  &  S:?.?  &  S:?.? & \\
\hline
\textcolor{blue}{\small 10} & P:?.?  &  P:?.?  &  P:?.?  &  P:?.?  &  P:?.?  &  P:?.?  &  P:?.?  &  P:?.?  &  P:?.?  &  P:?.?  &  P:?.?  &  P:?.?  &  P:?.?  &  P:?.?  &  P:?.?  &  P:?.?  &  P:?.?  &  P:?.?  &  P:?.?  &  P:?.? & \textcolor{blue}{\small 10}\\
 & S:?.?  &  S:?.?  &  S:?.?  &  S:?.?  &  S:?.?  &  S:?.?  &  S:?.?  &  S:?.?  &  S:?.?  &  S:?.?  &  S:?.?  &  S:?.?  &  S:?.?  &  S:?.?  &  S:?.?  &  S:?.?  &  S:?.?  &  S:?.?  &  S:?.?  &  S:?.? & \\
\hline
\textcolor{blue}{\small 9} & P:?.?  &  P:?.?  &  P:?.?  &  P:?.?  &  P:?.?  &  P:?.?  &  P:?.?  &  P:?.?  &  P:?.?  &  P:?.?  &  P:?.?  &  P:?.?  &  P:?.?  &  P:?.?  &  P:?.?  &  P:?.?  &  P:?.?  &  P:?.?  &  P:?.?  &  P:?.? & \textcolor{blue}{\small 9}\\
 & S:?.?  &  S:?.?  &  S:?.?  &  S:?.?  &  S:?.?  &  S:?.?  &  S:?.?  &  S:?.?  &  S:?.?  &  S:?.?  &  S:?.?  &  S:?.?  &  S:?.?  &  S:?.?  &  S:?.?  &  S:?.?  &  S:?.?  &  S:?.?  &  S:?.?  &  S:?.? & \\
\hline
\textcolor{blue}{\small 8} & P:?.?  &  P:?.?  &  P:?.?  &  P:?.?  &  P:?.?  &  P:?.?  &  P:?.?  &  P:?.?  &  P:?.?  &  P:?.?  &  P:?.?  &  P:?.?  &  P:?.?  &  P:?.?  &  P:?.?  &  P:?.?  &  P:?.?  &  P:?.?  &  P:?.?  &  P:?.? & \textcolor{blue}{\small 8}\\
 & S:?.?  &  S:?.?  &  S:?.?  &  S:?.?  &  S:?.?  &  S:?.?  &  S:?.?  &  S:?.?  &  S:?.?  &  S:?.?  &  S:?.?  &  S:?.?  &  S:?.?  &  S:?.?  &  S:?.?  &  S:?.?  &  S:?.?  &  S:?.?  &  S:?.?  &  S:?.? & \\
\hline
\textcolor{blue}{\small 7} & P:?.?  &  P:?.?  &  P:?.?  &  P:?.?  &  P:?.?  &  P:?.?  &  P:?.?  &  P:?.?  &  P:?.?  &  P:?.?  &  P:?.?  &  P:?.?  &  P:?.?  &  P:?.?  &  P:?.?  &  P:?.?  &  P:?.?  &  P:?.?  &  P:?.?  &  P:?.? & \textcolor{blue}{\small 7}\\
 & S:?.?  &  S:?.?  &  S:?.?  &  S:?.?  &  S:?.?  &  S:?.?  &  S:?.?  &  S:?.?  &  S:?.?  &  S:?.?  &  S:?.?  &  S:?.?  &  S:?.?  &  S:?.?  &  S:?.?  &  S:?.?  &  S:?.?  &  S:?.?  &  S:?.?  &  S:?.? & \\
\hline
\textcolor{blue}{\small 6} & P:?.?  &  P:?.?  &  P:?.?  &  P:?.?  &  P:?.?  &  P:?.?  &  P:?.?  &  P:?.?  &  P:?.?  &  P:?.?  &  P:?.?  &  P:?.?  &  P:?.?  &  P:?.?  &  P:?.?  &  P:?.?  &  P:?.?  &  P:?.?  &  P:?.?  &  P:?.? & \textcolor{blue}{\small 6}\\
 & S:?.?  &  S:?.?  &  S:?.?  &  S:?.?  &  S:?.?  &  S:?.?  &  S:?.?  &  S:?.?  &  S:?.?  &  S:?.?  &  S:?.?  &  S:?.?  &  S:?.?  &  S:?.?  &  S:?.?  &  S:?.?  &  S:?.?  &  S:?.?  &  S:?.?  &  S:?.? & \\
\hline
\textcolor{blue}{\small 5} & P:?.?  &  P:?.?  &  P:?.?  &  P:?.?  &  P:?.?  &  P:?.?  &  P:?.?  &  P:?.?  &  P:?.?  &  P:?.?  &  P:?.?  &  P:?.?  &  P:?.?  &  P:?.?  &  P:?.?  &  P:?.?  &  P:?.?  &  P:?.?  &  P:?.?  &  P:?.? & \textcolor{blue}{\small 5}\\
 & S:?.?  &  S:?.?  &  S:?.?  &  S:?.?  &  S:?.?  &  S:?.?  &  S:?.?  &  S:?.?  &  S:?.?  &  S:?.?  &  S:?.?  &  S:?.?  &  S:?.?  &  S:?.?  &  S:?.?  &  S:?.?  &  S:?.?  &  S:?.?  &  S:?.?  &  S:?.? & \\
\hline
\textcolor{blue}{\small 4} & P:?.?  &  P:?.?  &  P:?.?  &  P:?.?  &  P:?.?  &  P:?.?  &  P:?.?  &  P:?.?  &  P:?.?  &  P:?.?  &  P:?.?  &  P:?.?  &  P:?.?  &  P:?.?  &  P:?.?  &  P:?.?  &  P:?.?  &  P:?.?  &  P:?.?  &  P:?.? & \textcolor{blue}{\small 4}\\
 & S:?.?  &  S:?.?  &  S:?.?  &  S:?.?  &  S:?.?  &  S:?.?  &  S:?.?  &  S:?.?  &  S:?.?  &  S:?.?  &  S:?.?  &  S:?.?  &  S:?.?  &  S:?.?  &  S:?.?  &  S:?.?  &  S:?.?  &  S:?.?  &  S:?.?  &  S:?.? & \\
\hline
\textcolor{blue}{\small 3} & P:?.?  &  P:?.?  &  P:?.?  &  P:?.?  &  P:?.?  &  P:?.?  &  P:?.?  &  P:?.?  &  P:?.?  &  P:?.?  &  P:?.?  &  P:?.?  &  P:?.?  &  P:?.?  &  P:?.?  &  P:?.?  &  P:?.?  &  P:?.?  &  P:?.?  &  P:?.? & \textcolor{blue}{\small 3}\\
 & S:?.?  &  S:?.?  &  S:?.?  &  S:?.?  &  S:?.?  &  S:?.?  &  S:?.?  &  S:?.?  &  S:?.?  &  S:?.?  &  S:?.?  &  S:?.?  &  S:?.?  &  S:?.?  &  S:?.?  &  S:?.?  &  S:?.?  &  S:?.?  &  S:?.?  &  S:?.? & \\
\hline
\textcolor{blue}{\small 2} & P:?.?  &  P:?.?  &  P:?.?  &  P:?.?  &  P:?.?  &  P:?.?  &  P:?.?  &  P:?.?  &  P:?.?  &  P:?.?  &  P:?.?  &  P:?.?  &  P:?.?  &  P:?.?  &  P:?.?  &  P:?.?  &  P:?.?  &  P:?.?  &  P:?.?  &  P:?.? & \textcolor{blue}{\small 2}\\
 & S:?.?  &  S:?.?  &  S:?.?  &  S:?.?  &  S:?.?  &  S:?.?  &  S:?.?  &  S:?.?  &  S:?.?  &  S:?.?  &  S:?.?  &  S:?.?  &  S:?.?  &  S:?.?  &  S:?.?  &  S:?.?  &  S:?.?  &  S:?.?  &  S:?.?  &  S:?.? & \\
\hline
\textcolor{blue}{\small 1} & P:?.?  &  P:?.?  &  P:?.?  &  P:?.?  &  P:?.?  &  P:?.?  &  P:?.?  &  P:?.?  &  P:?.?  &  P:?.?  &  P:?.?  &  P:?.?  &  P:?.?  &  P:?.?  &  P:?.?  &  P:?.?  &  P:?.?  &  P:?.?  &  P:?.?  &  P:?.? & \textcolor{blue}{\small 1}\\
 & S:?.?  &  S:?.?  &  S:?.?  &  S:?.?  &  S:?.?  &  S:?.?  &  S:?.?  &  S:?.?  &  S:?.?  &  S:?.?  &  S:?.?  &  S:?.?  &  S:?.?  &  S:?.?  &  S:?.?  &  S:?.?  &  S:?.?  &  S:?.?  &  S:?.?  &  S:?.? & \\
\hline
\textcolor{blue}{\small 0} & P:?.?  &  P:?.?  &  P:?.?  &  P:?.?  &  P:?.?  &  P:?.?  &  P:?.?  &  P:?.?  &  P:?.?  &  P:?.?  &  P:?.?  &  P:?.?  &  P:?.?  &  P:?.?  &  P:?.?  &  P:?.?  &  P:?.?  &  P:?.?  &  P:?.?  &  P:?.? & \textcolor{blue}{\small 0}\\
 & S:?.?  &  S:?.?  &  S:?.?  &  S:?.?  &  S:?.?  &  S:?.?  &  S:?.?  &  S:?.?  &  S:?.?  &  S:?.?  &  S:?.?  &  S:?.?  &  S:?.?  &  S:?.?  &  S:?.?  &  S:?.?  &  S:?.?  &  S:?.?  &  S:?.?  &  S:?.? & \\
\hline
\hline
 & \textcolor{blue}{\small 19} & \textcolor{blue}{\small 18} & \textcolor{blue}{\small 17} & \textcolor{blue}{\small 16} & \textcolor{blue}{\small 15} & \textcolor{blue}{\small 14} & \textcolor{blue}{\small 13} & \textcolor{blue}{\small 12} & \textcolor{blue}{\small 11} & \textcolor{blue}{\small 10} & \textcolor{blue}{\small 9} & \textcolor{blue}{\small 8} & \textcolor{blue}{\small 7} & \textcolor{blue}{\small 6} & \textcolor{blue}{\small 5} & \textcolor{blue}{\small 4} & \textcolor{blue}{\small 3} & \textcolor{blue}{\small 2} & \textcolor{blue}{\small 1} & \textcolor{blue}{\small 0} & \\
\hline
\end{tabular}
}
%%   \end{center}
%%   \caption{Expected  printable   LIS  cabling  table  for   the  Italy
%%     calorimeter  main  wall  (front  view).  Question  marks  will  be
%%     replaced by the real fibers' identifiers when the final table will
%%     be  available. Similar  tables will  be provided  for X-walls  and
%%     gamma veto lines.}
%%   \label{fig:lis:sheet:1}
%% \end{figure}


\subsection{Labels}

A  Python  script is  provided  to  automatically generate,  from  the
CaloHV cabling table, lists of labels for all HV harnesses and
cables.  The labels must be stuck on the terminations of all harnesses
and  cables  to  help  the   cabling  team  to  identifiy  the  proper
connections  between  CAEN HV boards/external HV harnesses/patch
panel/internal HV cables/PMT.   Figure  \ref{fig:calohv:labels:1}
shows  where various  kinds  of labels  are supposed  to  be stuck  on
HV harnesses and cables.

\begin{figure}[h!]
  \begin{center}
    \scalebox{0.75}{\input{\pdftextimgpath/fig-calohv-labels-1.pdftex_t}}
  \end{center}
  \caption{CaloHV labelling of HV harnesses and internal cables}
  \label{fig:calohv:labels:1}
\end{figure}

%% end
